\documentclass{ximera}

% \pgfplotsset{compat=1.10}

%\input{preamble.tex}
%\prerequisites{Het bekijken van de webcolleges die horen bij les 2.}
%\outcomes{Begrip van krachten en evenwichten van een punt en dit kunnen toepassen op een eenvoudig mechanisch systeem.}
\title{Zelftesten les 2}
\begin{document}
\begin{abstract}
Dit zijn de zelftesten die je moet maken ter voorbereiding van de tweede les van Toegepaste Mechanica, deel 1.
\end{abstract}
\maketitle

\begin{question}
Een blok van 10 kg hangt in het aan een (massaloze) koord aan een haak A. 
Hoe groot is de spankracht in het touw (in N)?


\newcommand{\length}{2cm}
\newcommand{\angleDef}{30}
\newcommand{\width}{\length * 0.866}
\newcommand{\height}{\length *0.5}
\newcommand{\lengthBelow}{1cm}
\newcommand{\sideCube}{1cm}
\newcommand{\radius}{\length / 3}

% \centering
% \begin{tikzpicture}
%   \coordinate [label={above:$A$}] (A) at (0, 0);
%   \coordinate (B) at ($(A) + (0,-\lengthBelow)$);
%   
%   \coordinate (rect1) at ($(B)+(-\sideCube*0.5,0)$);
%   \coordinate (rect2) at ($(B)+(\sideCube*0.5,-\sideCube)$);
%   
%   \draw [fill=gray] (rect1) rectangle  (rect2);
%   
%   \draw [dashed] ($(A)+(-\width*0.5,0)$) -- ($(A)+(\width*0.5,0)$);
%  
%   \draw [very thick] (A) -- (B);
%   
%   \node[draw=none,align=center,label={below:10 kg}] at (B) {};
%   
% \end{tikzpicture}

\begin{hint}
 Maak een vrijlichaamsdiagramma
\begin{hint} 
 In \textbf{stap 1} stellen we een vrijlichaamsdiagram op in drie opeenvolgende stappen:
\begin{enumerate}
 \item \textbf{bepaling van het lichaam dat vrijgemaakt wordt:} \newline
 We maken het blok $b$ en de kabel $k$ vrij:
 
% \begin{center}
% \begin{tabular}{ccc}
% \begin{tikzpicture}
%   \coordinate (A) at (0, 0);
%   \coordinate (B) at ($(A) + (0,-\lengthBelow)$); 
%   \coordinate (rect1) at ($(B)+(-\sideCube*0.5,0)$);
%   \coordinate (rect2) at ($(B)+(\sideCube*0.5,-\sideCube)$);
%   \fill (B) circle [radius=2pt] node[above left] {B};
%    
%   \draw [fill=gray] (rect1) rectangle  (rect2);
%   \node[draw=none,align=center,label={below:10 kg}] at (B) {};
% \end{tikzpicture}
% &
% \begin{tikzpicture}
%   \coordinate (A) at (0, 0);
%   \coordinate (B) at ($(A) + (0,-\lengthBelow)$); 
%   \coordinate (rect1) at ($(B)+(-\sideCube*0.5,0)$);
%   \coordinate (rect2) at ($(B)+(\sideCube*0.5,-\sideCube)$);
%   \draw [very thick] (A) -- (B);
%   \fill (A) circle [radius=2pt] node[left] {A};
%   \fill (B) circle [radius=2pt] node[left] {B};
% 
% \end{tikzpicture}
% \end{tabular}
%  \end{center}
 
 \item \textbf{aanduiding van de niet-contactkrachten op het lichaam:}\newline
 In dit geval is er enkel het gewicht van het lichaam: $\vec G_b$:
 \begin{center}
%  \begin{tabular}{ccc}
% \begin{tikzpicture}
%   \coordinate (A) at (0, 0);
%   \coordinate (B) at ($(A) + (0,-\lengthBelow)$); 
%   \coordinate (rect1) at ($(B)+(-\sideCube*0.5,0)$);
%   \coordinate (rect2) at ($(B)+(\sideCube*0.5,-\sideCube)$);
%   \coordinate (centerOfMass) at ($(B)+(0,-\sideCube*0.5)$);
%   \fill (B) circle [radius=2pt] node[above left] {B};
%    
%   \draw [fill=gray] (rect1) rectangle  (rect2);
%    \draw[->,very thick,blue](centerOfMass) -- ($(centerOfMass)+(-90:\length*0.7)$) node[right] {$\vec G_b$};
%   \node[draw=none,align=center,label={below:10 kg}] at (B) {};
% \end{tikzpicture}
% &
% \begin{tikzpicture}
%   \coordinate (A) at (0, 0);
%   \coordinate (B) at ($(A) + (0,-\lengthBelow)$); 
%   \coordinate (rect1) at ($(B)+(-\sideCube*0.5,0)$);
%   \coordinate (rect2) at ($(B)+(\sideCube*0.5,-\sideCube)$);
% 
%   \draw [very thick] (A) -- (B);
%   \fill (A) circle [radius=2pt] node[left] {A};
%   \fill (B) circle [radius=2pt] node[left] {B};
% 
% \end{tikzpicture}
% \end{tabular}
\end{center}
 
 \item \textbf{aanduiding van alle contactkrachten op het lichaam:} \newline

 \begin{center}
 \begin{tabular}{p{0.4\textwidth}p{0.4\textwidth}}
  In dit geval is het koord weggelaten. 
 Het weggelaten koord oefent een opwaartse kracht $\vec S_{bk}$ uit op het vrijgemaakte blok:
%  \begin{center}
% \begin{tikzpicture}
%   \coordinate (A) at (0, 0);
%   \coordinate (B) at ($(A) + (0,-\lengthBelow)$); 
%   \coordinate (rect1) at ($(B)+(-\sideCube*0.5,0)$);
%   \coordinate (rect2) at ($(B)+(\sideCube*0.5,-\sideCube)$);
%   \coordinate (centerOfMass) at ($(B)+(0,-\sideCube*0.5)$);
%   \fill (B) circle [radius=2pt] node[above left] {B};
%    
%   \draw [fill=gray] (rect1) rectangle  (rect2);
%   \draw[->,very thick,blue](centerOfMass) -- ($(centerOfMass)+(-90:\length*0.7)$) node[right] {$\vec G_b$};
%   \node[draw=none,align=center,label={below:10 kg}] at (B) {};
%   \draw[->,very thick,red](B) -- ($(B)+(90:\length*0.7)$) node[right] {$\vec F_{bk}$};
% \end{tikzpicture}
% \end{center}
&
  In dit geval is het blok en de haak weggelaten. 
 Het weggelaten blok oefent een neerwaartse kracht $\vec S_{kb}$ uit op de vrijgemaakt kabel. 
 De weggelaten haak oefent een opwaartse kracht $\vec S_{k0}$ uit op de vrijgemaakte kabel. 
%  \begin{center}
% \begin{tikzpicture}
%   \coordinate (A) at (0, 0);
%   \coordinate (B) at ($(A) + (0,-\lengthBelow)$); 
%   \coordinate (rect1) at ($(B)+(-\sideCube*0.5,0)$);
%   \coordinate (rect2) at ($(B)+(\sideCube*0.5,-\sideCube)$);
% 
%   \draw [very thick] (A) -- (B);
%   \fill (A) circle [radius=2pt] node[left] {A};
%   \fill (B) circle [radius=2pt] node[left] {B};
%   \draw[->,very thick,red](B) -- ($(B)+(-90:\length*0.7)$) node[right] {$\vec F_{kb}$};
%   \draw[->,very thick,red](A) -- ($(A)+(90:\length*0.7)$) node[right] {$\vec F_{k0}$};
% \end{tikzpicture}
% \end{center}
\end{tabular}
\end{center}
 \begin{itemize}
  \item het blok heeft een gewicht $\vec G_b = -  \left\{\begin{array}{c} 0 \\ 0 \\ -100N \end{array}\right\}$
  \item de wet van actie en reactie zegt dat: $\vec F_{kb} = -\vec F_{kb}  $ met $\vec F_{kb} $ is onbekend.
  \item de kracht van de haak op de kabel $\vec F_{k0}$ is nog onbekend.
 \end{itemize}

\end{enumerate}

\end{hint}
\end{hint}
\begin{hint}
  Stel het krachtenevenwicht op van het blok en de kabel.
\begin{hint}
 In \textbf{stap 2} stellen we het krachtenevenwicht op van het blok en de kabel:

 \begin{tabular}{p{0.45\textwidth}|p{0.45\textwidth}}
 \textbf{blok}:\newline
 \begin{equation*}
 \vec{F}_{\textrm{resulterend}} = \vec G_b +  \vec F _{bk} = \vec 0
\end{equation*}
Hieruit halen we de kabelkracht:
\begin{equation*}
 \vec F _{bk} =  - \vec G_b =  \left\{\begin{array}{c} 0 \\ 0 \\ 100N \end{array}\right\}
\end{equation*}
&
 \textbf{kabel}:\newline
 \begin{equation*}
 \vec{F}_{\textrm{resulterend}} =  \vec F _{kb} + \vec F _{k0} = \vec 0
\end{equation*}
Hieruit halen we de kracht van de haak op de kabel:
\begin{equation*}
  \vec F _{k0} = -\vec F _{kb} =   \vec F _{bk} =  \left\{\begin{array}{c} 0 \\ 0 \\ 100N \end{array}\right\}
\end{equation*}
 \end{tabular}

De kabel ondervindt dus een trekkracht van ...
%  \begin{center}
%  \begin{tabular}{p{0.4\textwidth}p{0.4\textwidth}}
%  \begin{center}
% \begin{tikzpicture}
%   \coordinate (A) at (0, 0);
%   \coordinate (B) at ($(A) + (0,-\lengthBelow)$); 
%   \coordinate (rect1) at ($(B)+(-\sideCube*0.5,0)$);
%   \coordinate (rect2) at ($(B)+(\sideCube*0.5,-\sideCube)$);
%   \coordinate (centerOfMass) at ($(B)+(0,-\sideCube*0.5)$);
%   \fill (B) circle [radius=2pt] node[above left] {B};
%    
%   \draw [fill=gray] (rect1) rectangle  (rect2);
%   \draw[->,very thick,blue](centerOfMass) -- ($(centerOfMass)+(-90:\length*0.7)$) node[right] {$\vec G_b$};
%   \node[draw=none,align=center,label={below:10 kg}] at (B) {};
%   \draw[->,very thick,red](B) -- ($(B)+(90:\length*0.7)$) node[right] {$\vec F_{bk}$};
% \end{tikzpicture}
% \end{center}
% &
% 
%  \begin{center}
% \begin{tikzpicture}
%   \coordinate (A) at (0, 0);
%   \coordinate (B) at ($(A) + (0,-\lengthBelow)$); 
%   \coordinate (rect1) at ($(B)+(-\sideCube*0.5,0)$);
%   \coordinate (rect2) at ($(B)+(\sideCube*0.5,-\sideCube)$);
% 
%   \draw [very thick] (A) -- (B);
%   \fill (A) circle [radius=2pt] node[left] {A};
%   \fill (B) circle [radius=2pt] node[left] {B};
%   \draw[->,very thick,red](B) -- ($(B)+(-90:\length*0.7)$) node[right] {$\vec F_{kb}$};
%   \draw[->,very thick,red](A) -- ($(A)+(90:\length*0.7)$) node[right] {$\vec F_{k0}$};
% \end{tikzpicture}
% \end{center}
% \end{tabular}
% \end{center}
 \end{hint}

 
\end{hint}

\answer{$100$} N

\end{question}


\end{document}
